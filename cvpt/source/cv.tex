%!TEX TS-program = xelatex
\documentclass[]{friggeri-cv}
\addbibresource{bibliography.bib}

\begin{document}
\header{Rafael Mariottini Tomazela}{}


% In the aside, each new line forces a line break
\begin{aside}
  \section{sobre}
    \vspace{0.30cm}
Rua Atanazio Soares
1043, Casa 14A
18075-000
Sorocaba/SP
~
Email: \href{mailto:rafael.mtomazela
@gmail.com}{rafael.mtomazela@
gmail.com}
~
LinkedIn: br.linkedin.com/
in/rafaeltomazela
~
Idade: 22
  \section{linguagens}
    \vspace{0.30cm}
    Português nativo, 
    Inglês fluente
  \section{competências}
    \vspace{0.30cm}
    Python, Ruby,
C/C++, Java, C\#,
Git/Subversion,
HTML/CSS básico,
SQL, Latex,
Linux
\end{aside}

%\section{interests}

%complex networks, social networks, community detection, community structure,
%overlapping communities, information diffusion, viral marketing, social
%inference, recommendation, data mining

%\section{Objetivo}

%Atuar na área de desenvolvimento de software trabalhando em projetos desafiadores e utilizando novas tecnologias.
\section{Experiência profissional}

\begin{entrylist}
  \entry
    {02-07 2015}
    {CESAR}
    {Estágio}
    {Desenvolvimento de um software de automação para um quadricóptero utilizando microcontroladores \emph{Propeller}. Ele foi automatizado usando dados de três sensores (acelerômetro, magnetômetro e giroscópio) e controladores PID para a ativação dos motores.}
    %{Desenvolvimento de um software para a automação de um quadricóptero em microcontroladores Propeller.
    
    %\vspace{0.1cm}
    %Implementei 3 sensores: acelerômetro, magnetômetro e giroscópio, e a fusão destes. Desenvolvi também a ativação dos motores utilizando controladores PID.}
    
  \entry
    {05-08 2014}
    {Little Guy Games, Toronto}
    {Estágio de verão}
    {Desenvolvimento de 3 jogos usando Unity 3D com C\#: trabalho extensivo no jogo \emph{Yu-Gi-Oh! Duel Dash} e contribuições menores em outros dois projetos não anunciados.}
    %{Desenvolvimento de 3 jogos utilizando Unity 3D com C\#. 
    
    %\vspace{0.1cm}
    %Trabalhei no editor e nas partes lógicas e gráficas do jogo \emph{Yu-Gi-Oh! Duel Dash}, que foi desenvolvido em conjunto com outro programador para um evento da Konami ocorrido em 2014. Trabalhei também programando outros dois jogos não anunciados.}
\end{entrylist}



\section{Educação}

\begin{entrylist}
  \entry
    {2014}
    {Bacharelado em Ciências da Computação}{Graduação sanduíche}
    {University of Toronto}
  \entry
    {2011-2015}
    {Bacharelado em Ciências da Computação}{}
    {Universidade Federal de São Carlos}
\end{entrylist}


\section{Iniciação Científica}

\begin{entrylist}

\entry
    {2012-2013}
    {Implementação e avaliação de um algoritmo de agrupamento paralelo no MOCLE}{}
    {Supervisionado pela professora Dra. Tiemi Cristine Sakata}
\entry 
    {2013-2015}
{Estudo e desenvolvimento da arquitetura distribuída para o DataExplorer}{}
{Supervisionado pela professora Dra. Katti Faceli}

\end{entrylist}
  

\section{Outras atividades}



\begin{entrylist}

\entry{2013} 
{Participação na fase latino-america da maratona de programação da ACM-ICPC}{}{}


\entry{2011-2013 }
{Participação na maratona de programação anual da Universidade Federal de São Carlos}{}{}

\entry{2011-2013 } 
{Participação na fase regional da maratona de programação da ACM-ICPC}{}{}



\end{entrylist}



\section{Cursos Online}
\begin{entrylist}
\entry
{\phantom{}2012}
{Inteligência Artificial}
{Por Dan Klein}
{The University of California at Berkeley}

\entry
{\phantom{}2012}
{Software as a Service}
{Por Armando Fox e David Patterson}
{The University of California at Berkeley}

\entry
{\phantom{}2011}
{Introdução à Inteligência Artificial}
{Por Peter Norvig e Sebastian Thrun}
{Stanford University}

\entry
{\phantom{}2011}
{Aprendizado de Máquina}
{Por Andrew Ng}
{Stanford University}

\end{entrylist}


\end{document}
